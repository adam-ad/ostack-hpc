
diskimage-builder or dib uses environment variables and elements to customize the images. This section setups default environment variable to build HPC images. For debugging purpose, we will create default user chpc with a password intel8086, with sudo privilege. These variables are used by element devuser. 

we will create function setup\_dib\_hpc\_base.

%% we will create function  setup_dib_hpc_base. 
% begin_ohpc_run
% ohpc_validation_newline
%% ohpc_validation_comment # SECTION BMNOS
% ohpc_validation_comment function to setup dib environment setup_dib_hpc_base
% ohpc_validation_comment
\begin{lstlisting}[language=bash,keywords={}]
[ctrlr](*\#*) function setup_dib_hpc_base() {
\end{lstlisting}
% end_ohpc_run

Install dib if it is not already installed by calling setup\_dib function, which we created in previous section

% begin_ohpc_run
% ohpc_validation_comment   Install dib if it is not already installed
\begin{lstlisting}[language=bash,keywords={}]
[ctrlr](*\#*)     setup_dib
\end{lstlisting}
% end_ohpc_run

% begin_ohpc_run
% ohpc_validation_comment   diskimage-builder initial config
\begin{lstlisting}[language=bash,keywords={}]
[ctrlr](*\#*)     export DIB_DEV_USER_USERNAME=chpc
[ctrlr](*\#*)     export DIB_DEV_USER_PASSWORD=intel8086
[ctrlr](*\#*)     export DIB_DEV_USER_PWDLESS_SUDO=1
\end{lstlisting}
% end_ohpc_run

Now add path to custom elements which are not part of base diskimage-builder. OpenHPC provides few HPC elements. [Note: This also can be part of openHPC provided rpm for dib. In that case remove this step]

% begin_ohpc_run
%% ohpc_validation_comment # SECTION BMNOS
% ohpc_validation_comment   Add custom elements in DIB
% ohpc_validation_comment
\begin{lstlisting}[language=bash,keywords={}]
[ctrlr](*\#*)     export ELEMENTS_PATH="$(realpath ../../dib/hpc/elements)"
\end{lstlisting}
% end_ohpc_run

Add path to HPC specific files [note: same as earlier, this too can be part of rpm package]

% begin_ohpc_run
%% ohpc_validation_comment # SECTION BMNOS
% ohpc_validation_comment   Add path to HPC specific files
% ohpc_validation_comment
\begin{lstlisting}[language=bash,keywords={}]
[ctrlr](*\#*)     export DIB_HPC_FILE_PATH="$(realpath ../../dib/hpc/hpc-files/)"
\end{lstlisting}
% end_ohpc_run

HPC elements are common for OpenHPC and Intel HPC Orchestrator, environment variable "DIB\_HPC\_BASE" tell dib which one to pick. For OpenHPC set environment variable

% begin_ohpc_run
%% ohpc_validation_comment # SECTION BMNOS
% ohpc_validation_comment   Set DIB Element
% ohpc_validation_comment
\begin{lstlisting}[language=bash,keywords={}]
[ctrlr](*\#*)     export DIB_ HPC_BASE="ohpc"
\end{lstlisting}
% end_ohpc_run

Make sure open hpc packages is installed. ohpc\_pkg is one of the input setup earlier in this document.

% begin_ohpc_run
%% ohpc_validation_comment # SECTION BMNOS
% ohpc_validation_comment   Check package installed
% ohpc_validation_comment
\begin{lstlisting}[language=bash,keywords={}]
[ctrlr](*\#*)     yum -y install ${ohpc_pkg}
\end{lstlisting}
% end_ohpc_run

Export same to DIB.

% begin_ohpc_run
%% ohpc_validation_comment # SECTION BMNOS
% ohpc_validation_comment   Export to DIB
% ohpc_validation_comment
\begin{lstlisting}[language=bash,keywords={}]
[ctrlr](*\#*)     export DIB_HPC_OHPC_PKG=${ohpc_pkg}
\end{lstlisting}
% end_ohpc_run

Create list of HPC elements needed to build HPC images, by starting hpc-env-base. This element will setup basic hpc environment to build hpc images.

% begin_ohpc_run
%% ohpc_validation_comment # SECTION BMNOS
% ohpc_validation_comment    Create listof elements
% ohpc_validation_comment
\begin{lstlisting}[language=bash,keywords={}]
[ctrlr](*\#*)     DIB_HPC_ELEMENTS="hpc-env-base"
[ctrlr](*\#*) } # end of the function
\end{lstlisting}
% end_ohpc_run
