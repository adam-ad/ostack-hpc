% begin_ohpc_run
% ohpc_comment_header Allow for optional sleep to wait for provisioning to complete
% ohpc_command sleep ${provision_wait}
% end_ohpc_run

In section , the \rms{} workload manager was installed
and configured for use on both the {\em master} host and {\em compute} node
instances. With the cluster nodes up and functional, we can now startup the
resource manager services on the {\em master} host in preparation for running 
user jobs.

The following command can be used to startup the necessary services to support
resource management under \rms{}.

% begin_ohpc_run
% ohpc_comment_header Resource Manager Startup \ref{sec:rms_startup}
\begin{lstlisting}[language=bash,keywords={}]
# start pbspro daemons on master host
[sms](*\#*) systemctl enable pbs
[sms](*\#*) systemctl start pbs
\end{lstlisting}
% end_ohpc_run

In the default configuration, the {\em compute} hosts have not been added to
the \rms{} resource manager. To add the {\em compute} hosts, execute the
following to register each host and apply several key global configuration
options.

% begin_ohpc_run
\begin{lstlisting}[language=bash,keywords={}]
# initialize PBS path
[sms](*\#*) . /etc/profile.d/pbs.sh
# enable user environment propagation (needed for modules support)
[sms](*\#*) qmgr -c "set server default_qsub_arguments= -V"
# enable uniform multi-node MPI task distribution
[sms](*\#*) qmgr -c "set server resources_default.place=scatter"
# register compute hosts with pbspro
[sms](*\#*) for host in "${c_name[@]}"; do
           qmgr -c "create node $host"
        done
\end{lstlisting}
% end_ohpc_run

