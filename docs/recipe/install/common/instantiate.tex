	To instantiate OpenHPC system, we will need to prepare Openstack components with HPC images, networking, and other relevant configurations. After the configuration we will instantiate HPC head node and HPC compute node using nova. 

	It is assumed that you have a working configuration of OpenStack, with controller and network services (i.e. keystone, nova, ironic, glance, neutron, mongodb, rabbitmq server, heat etc). 

	The controller node should be configured with OpenVSwitch Bridge on an internal network port. If your configuration is different, adjust these settings accordingly, but keep in mind, YMMV.

	Two tenants, named \texttt{admin} and \texttt{services} are created in keystone to manage the services. All the services are created by system admin. 

	Below is the expected endpoint list.

% begin_ohpc_run
% ohpc_validation_comment #   XFILEX
% end_ohpc_run

% begin_ohpc_run
% ohpc_command #!/bin/bash
% ohpc_validation_comment #   FILE: deploy_chpc_openstack part 1
% end_ohpc_run

\begin{lstlisting}[language=bash,keywords={}]
[ctrlr](*\#*) openstack service list
+----------------------------------+-----------+---------------+---------------+
| ID                                         | Region    | Service Name  | Service Type  |
+----------------------------------+-----------+---------------+---------------+
| d5aeeb54713745c29ed3c2e4a97f59bd | RegionOne | ironic        | baremetal     |
| 86c71badbf8b4446a1b699eef05f3f41 | RegionOne | nova          | compute       |
| 70d138db26214d0bbc6b3ade8bf6f6f8 | RegionOne | gnocchi       | metric        |
| f34c3a58b9c648aaacabeeefd589a0d2 | RegionOne | neutron       | network       |
| 789c3fb6f9ae4e249ee4023484ccb5fc | RegionOne | aodh          | alarming      |
| 2531392e2d084b4582b364572e79a7b5 | RegionOne | heat          | orchestration |
| c183b73f654e454eaf5784c4b98149d8 | RegionOne | Image Service | image         |
| 850b3c2943df4dca99338ff2013f657b | RegionOne | cinder        | volume        |
| 81cefa79212a4780abe5a1da281a0172 | RegionOne | novav3        | computev3     |
| 36a6a7c7968a4a94bea07c8e30fa5c4b | RegionOne | keystone      | identity      |
| db70a8676dd44dd09b3ada7475e67383 | RegionOne | cinderv3      | volumev3      |
| c4161d4c9c6b4080b2cb66c2f580853d | RegionOne | ceilometer    | metering      |
| d5714e8adb094671ad0388d04214c44d | RegionOne | cinderv2      | volumev2      |
+----------------------------------+-----------+---------------+---------------+
[ctrlr](*\#*) openstack project list
+----------------------------------+----------+
| ID                               | Name     |
+----------------------------------+----------+
| 7464fcc8f1b34048bd09fe165d18647b | admin    |
| b1ed7efb53cc44c8b06daaee15b6a296 | services |
+----------------------------------+----------+
\end{lstlisting}


	The recipe that follows is tested with a controller node installed and configured using RDO PackStack.
	The reference section on the RDO/Packstack website {\em www.rdoproject.org }can provide more detail on PackStack installation of Openstack.
	Before starting with HPC instantiation, please export the Openstack credentials, as we will be using them during Openstack configuration. 

	You can do this manually, as such:

\begin{lstlisting}[language=bash,keywords={}]
[ctrlr](*\#*) unset OS_SERVICE_TOKEN
[ctrlr](*\#*) export OS_USERNAME=admin
[ctrlr](*\#*) export OS_PASSWORD=<>
[ctrlr](*\#*) export OS_AUTH_URL=<>
[ctrlr](*\#*) export PS1='[\u@\h \W(keystone_admin)]\$ '
[ctrlr](*\#*) export OS_TENANT_NAME=admin
[ctrlr](*\#*) export OS_REGION_NAME=<>  
\end{lstlisting}

	OR, if you've deployed via PackStack, you can just source the keystonerc\_admin file. These credentials will be used throughout the rest of this document. If you have an existing Openstack installation with more complex credentials, you will need to set them per your configuration.

% begin_ohpc_run
\begin{lstlisting}[language=bash,keywords={}]
[ctrlr](*\#*) source ${HOME}/keystonerc_admin
\end{lstlisting}
% end_ohpc_run
