At this point, the {\em master} server should be able to boot the newly defined
compute nodes. Assuming that the compute node BIOS settings are configured to
boot over PXE, all that is required to initiate the provisioning process is to
power cycle each of the desired hosts using IPMI access.
The following commands use the \texttt{ipmitool} utility to initiate power
resets on each of the four compute hosts. Note that the utility requires that
the \texttt{IPMI\_PASSWORD} environment variable be set with the local BMC password in
order to work interactively.

% begin_ohpc_run
% ohpc_comment_header Boot compute nodes \ref{sec:boot_computes}
\begin{lstlisting}[language=bash,keywords={},upquote=true]
[sms](*\#*) for ((i=0; i<${num_computes}; i++)) ; do
              ipmitool -E -I lanplus -H ${c_bmc[$i]} -U ${bmc_username} chassis power reset
        done
\end{lstlisting} 
% end_ohpc_run

Once kicked off, the boot process should take less than 5 minutes (depending on
BIOS post times) and you can verify that the compute hosts are available via
ssh, or via parallel ssh tools to multiple hosts. For example, to run a command
on the newly imaged compute hosts using \texttt{pdsh}, execute the following:

%\iftoggle{isCentOS_ww_pbs_x86}{\clearpage}
%\iftoggle{isCentOS}{\clearpage}
  
\begin{lstlisting}[language=bash]
[sms](*\#*) pdsh -w c[1-4] uptime
c1  05:03am  up   0:02,  0 users,  load average: 0.20, 0.13, 0.05
c2  05:03am  up   0:02,  0 users,  load average: 0.20, 0.14, 0.06
c3  05:03am  up   0:02,  0 users,  load average: 0.19, 0.15, 0.06
c4  05:03am  up   0:02,  0 users,  load average: 0.15, 0.12, 0.05
\end{lstlisting}
