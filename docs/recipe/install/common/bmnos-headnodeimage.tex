To build head node (aka sms) images, we need to install server packages of HPC components. This is accomplished by setting image type to sms. Default image type in hpc elements is “compute”. we will create a function function prepare\_sms\_image

% begin_ohpc_run
% ohpc_validation_newline
% ohpc_validation_comment function to prepare head node
% ohpc_validation_comment
\begin{lstlisting}[language=bash,keywords={}]
[ctrlr](*\#*) function prepare_sms_image() {
\end{lstlisting} 
% end_ohpc_run

First check if user has requested to create images by verifying environment variables chpc\_create\_new\_image and chpc\_image\_sms. If user already supplied images then we will copy images to our common image location and setup environment variable for later use
% begin_ohpc_run
\begin{lstlisting}[language=bash,keywords={}]
[ctrlr](*\#*)     if [[ ${chpc_create_new_image} -ne 1 ]] && [[ -s $chpc_image_sms ]]; then
[ctrlr](*\#*)         # No need to create an image, image is provided by user
[ctrlr](*\#*)         echo -n "Skiping cloud sms-image build, Image provided:"
[ctrlr](*\#*)         echo "$chpc_image_sms"
[ctrlr](*\#*)         CHPC_IMAGE_DEST=$CHPC_CLOUD_IMAGE_PATH/$(basename $chpc_image_sms)
[ctrlr](*\#*)         if [[ ! -e $CHPC_IMAGE_DEST ]]; then
[ctrlr](*\#*)             sudo cp $chpc_image_sms $CHPC_CLOUD_IMAGE_PATH
[ctrlr](*\#*)         fi
[ctrlr](*\#*)         chpc_image_sms=$CHPC_IMAGE_DEST
[ctrlr](*\#*)     else
\end{lstlisting} 
% end_ohpc_run

. If user has not supplied images then we will build sms image here. disk-image-builder will supports two type of HPC images, sms and compute.  

% begin_ohpc_run
\begin{lstlisting}[language=bash,keywords={}]
[ctrlr](*\#*)         # setup environment varioable to indicate sms image type
[ctrlr](*\#*)         export DIB\_HPC\_IMAGE\_TYPE=sms
\end{lstlisting} 
% end_ohpc_run

Enable SLURM resource manager for head node.

% begin_ohpc_run

\begin{lstlisting}[language=bash,keywords={}]
[ctrlr](*\#*)         DIB\_HPC\_ELEMENTS+=" hpc-slurm"
\end{lstlisting} 
 % end_ohpc_run

Add optional OpenHPC Components

% begin_ohpc_run

\begin{lstlisting}[language=bash,keywords={}]
[ctrlr](*\#*)         if [[ \$\{enable\_mrsh\} -eq 1 ]];then
[ctrlr](*\#*)             DIB\_HPC\_ELEMENTS+=" hpc-mrsh"
[ctrlr](*\#*)         fi
\end{lstlisting} 
 % end_ohpc_run

We will also setup HPC development environment on HPC head node. 
Enable gnu compiler

% begin_ohpc_run

\begin{lstlisting}[language=bash,keywords={}]
[ctrlr](*\#*)         export DIB\_HPC\_COMPILER="gnu"
\end{lstlisting} 
 % end_ohpc_run

Enable openmpi \& mvapich2

% begin_ohpc_run

\begin{lstlisting}[language=bash,keywords={}]
[ctrlr](*\#*)         export DIB\_HPC\_MPI="openmpi mvapich2"
\end{lstlisting} 
 % end_ohpc_run

Enable performance tools

% begin_ohpc_run

\begin{lstlisting}[language=bash,keywords={}]
[ctrlr](*\#*)         export DIB\_HPC\_PERF\_TOOLS="perf-tools"
\end{lstlisting} 
 % end_ohpc_run

Enable 3rd party libraries serial-libs, parallel-libs, io-libs, python-libs and runtimes

% begin_ohpc_run

\begin{lstlisting}[language=bash,keywords={}]
[ctrlr](*\#*)         export DIB\_HPC\_3RD\_LIBS="serial-libs parallel-libs io-libs python-libs runtimes"
\end{lstlisting} 
 % end_ohpc_run

Add hpc development environment element to list of elements

% begin_ohpc_run

\begin{lstlisting}[language=bash,keywords={}]
[ctrlr](*\#*)         DIB\_HPC\_ELEMENTS+=" hpc-dev-env"
\end{lstlisting} 
 % end_ohpc_run

Now create a sms image with element local-config, dhcp-all-interfaces, devuser, selinux-permisive and all hpc specific elements. Element local-config copies your local environment into image, which is the local users, their password and permissions. Element devuser will create new user specified by environment variable "DIB\_DEV\_USER\_USERNAME". 

% begin_ohpc_run

\begin{lstlisting}[language=bash,keywords={}]
[ctrlr](*\#*)         disk-image-create centos7 vm local-config dhcp-all-interfaces devuser \
         selinux-permissive \$DIB\_HPC\_ELEMENTS -o icloud-hpc-cent7-sms
\end{lstlisting} 
 % end_ohpc_run

It will take a while to build an image. Once image is built copy it to standard openHPC path.

% begin_ohpc_run

\begin{lstlisting}[language=bash,keywords={}]
[ctrlr](*\#*)         chpc_image_sms="$( realpath icloud-hpc-cent7.qcow2)"
[ctrlr](*\#*)         mkdir -p $CHPC_CLOUD_IMAGE_PATH
[ctrlr](*\#*)         mv -f \$chpc\_image\_sms \$CHPC\_CLOUD\_IMAGE\_PATH
[ctrlr](*\#*)         chpc\_image\_sms=\$CHPC\_CLOUD\_IMAGE\_PATH/\$(basename \$chpc\_image\_sms)
[ctrlr](*\#*)     fi # end of else of or if
[ctrlr](*\#*) } # end of function
\end{lstlisting} 
 % end_ohpc_run
