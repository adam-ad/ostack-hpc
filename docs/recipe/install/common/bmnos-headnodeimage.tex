	Building head node (SMS) images requires installing the server packages of HPC components into the image. This is accomplished by setting the image type to "sms". The default image type in HPC elements is "compute". We will create a function prepare\_sms\_image to build this image, which will be called later in the document.

% begin_ohpc_run
% ohpc_validation_newline
% ohpc_validation_comment function to prepare head node
% ohpc_validation_comment
\begin{lstlisting}[language=bash,keywords={}]
[ctrlr](*\#*) function prepare_sms_image() {
\end{lstlisting} 
% end_ohpc_run

	Check if the user has requested to create images by verifying environment variables chpc\_create\_new\_image and chpc\_image\_sms. If the user has supplied images then we will copy images to our common image location and setup environment variables for later use.
	
% begin_ohpc_run
\begin{lstlisting}[language=bash,keywords={}]
[ctrlr](*\#*)     if [[ ${chpc_create_new_image} -ne 1 ]] && [[ -s $chpc_image_sms ]]; then
[ctrlr](*\#*)         # No need to create an image, image is provided by the user
[ctrlr](*\#*)         echo -n "Skipping cloud sms-image build, image provided:"
[ctrlr](*\#*)         echo "$chpc_image_sms"
[ctrlr](*\#*)         CHPC_IMAGE_DEST=$CHPC_CLOUD_IMAGE_PATH/$(basename $chpc_image_sms)
[ctrlr](*\#*)         if [[ ! -e $CHPC_IMAGE_DEST ]]; then
[ctrlr](*\#*)             sudo cp $chpc_image_sms $CHPC_CLOUD_IMAGE_PATH
[ctrlr](*\#*)         fi
[ctrlr](*\#*)         chpc_image_sms=$CHPC_IMAGE_DEST
[ctrlr](*\#*)     else
\end{lstlisting} 
% end_ohpc_run

	If the user has not supplied images then we will build the sms image here. Disk-image-builder supports two types of HPC images, "sms" and "compute".  

% begin_ohpc_run
\begin{lstlisting}[language=bash,keywords={}]
[ctrlr](*\#*)         # setup environment variable to indicate sms image type
[ctrlr](*\#*)         setup_dib_hpc_base
[ctrlr](*\#*)         export DIB_HPC_IMAGE_TYPE=sms
\end{lstlisting} 
% end_ohpc_run

	Enable HPC resource manager for "head" node.

% begin_ohpc_run

\begin{lstlisting}[language=bash,keywords={}]
[ctrlr](*\#*)         DIB_HPC_ELEMENTS+=" hpc-slurm"
\end{lstlisting} 
 % end_ohpc_run

	Add optional OpenHPC components.

% begin_ohpc_run

\begin{lstlisting}[language=bash,keywords={}]
[ctrlr](*\#*)         if [[ ${enable_mrsh} -eq 1 ]];then
[ctrlr](*\#*)             DIB_HPC_ELEMENTS+=" hpc-mrsh"
[ctrlr](*\#*)         fi
\end{lstlisting} 
 % end_ohpc_run

	We will also setup an HPC development environment on the HPC head node. 

	Start by enabling the gnu compiler on head node.

% begin_ohpc_run

\begin{lstlisting}[language=bash,keywords={}]
[ctrlr](*\#*)         export DIB_HPC_COMPILER="gnu7"
\end{lstlisting} 
 % end_ohpc_run

	Enable openmpi \& mvapich2.

% begin_ohpc_run

\begin{lstlisting}[language=bash,keywords={}]
[ctrlr](*\#*)         export DIB_HPC_MPI="openmpi mvapich2"
\end{lstlisting} 
 % end_ohpc_run

	Enable performance tools.

% begin_ohpc_run

\begin{lstlisting}[language=bash,keywords={}]
[ctrlr](*\#*)         export DIB_HPC_PERF_TOOLS="perf-tools"
\end{lstlisting} 
 % end_ohpc_run

	Enable 3rd party libraries serial-libs, parallel-libs, io-libs, python-libs, and runtimes.

% begin_ohpc_run

\begin{lstlisting}[language=bash,keywords={}]
[ctrlr](*\#*)         export DIB_HPC_3RD_LIBS="serial-libs parallel-libs io-libs python-libs runtimes"
\end{lstlisting} 
 % end_ohpc_run

	Add the HPC development environment element to the list of elements.

% begin_ohpc_run

\begin{lstlisting}[language=bash,keywords={}]
[ctrlr](*\#*)         DIB_HPC_ELEMENTS+=" hpc-dev-env"
\end{lstlisting} 
 % end_ohpc_run

	Now create an sms image with the following elements: local-config, dhcp-all-interfaces, devuser, selinux-permisive, and all HPC specific elements. Element local-config copies your local environment into the image, which is the local users, their password, and permissions. Element devuser will create a new user specified by environment variable "DIB\_DEV\_USER\_USERNAME". The image will be named  {\em  icloud-hpc-cent7-sms}.

% begin_ohpc_run

\begin{lstlisting}[language=bash,keywords={}]
[ctrlr](*\#*)         disk-image-create centos7 vm local-config dhcp-all-interfaces \
[ctrlr](*\#*)         	devuser selinux-permissive $DIB_HPC_ELEMENTS -o icloud-hpc-cent7-sms
\end{lstlisting} 
 % end_ohpc_run

	It will take awhile to build an image. Once the image is built, copy it to the standard OpenHPC path.

% begin_ohpc_run

\begin{lstlisting}[language=bash,keywords={}]
[ctrlr](*\#*)         chpc_image_sms="$( realpath icloud-hpc-cent7-sms.qcow2)"
[ctrlr](*\#*)         mkdir -p $CHPC_CLOUD_IMAGE_PATH
[ctrlr](*\#*)         mv -f $chpc_image_sms $CHPC_CLOUD_IMAGE_PATH
[ctrlr](*\#*)         chpc_image_sms=$CHPC_CLOUD_IMAGE_PATH/$(basename $chpc_image_sms)
[ctrlr](*\#*)     fi # end of else of or if
[ctrlr](*\#*) } # end of function
\end{lstlisting} 
 % end_ohpc_run
