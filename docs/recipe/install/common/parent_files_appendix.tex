# This file generates the parent scripts form the parse_doc

% begin_ohpc_run
% ohpc_command #  PFILEP
% ohpc_command ## FILE: ./common_functions
% ohpc_command #!/bin/bash
% ohpc_command 
% ohpc_command #validateIpInput () {
% ohpc_command #}
% ohpc_command function validateInputFile () {
% ohpc_command     if [[ -z "${inputFile}" || ! -e "${inputFile}" ]];then
% ohpc_command       echo "Error: Unable to access local input file -> \"${inputFile}\""
% ohpc_command       exit 1
% ohpc_command     else
% ohpc_command       . ${inputFile} || { echo "Error sourcing ${inputFile}"; exit 1; }
% ohpc_command     fi
% ohpc_command     _BADCOUNT=0
% ohpc_command     
% ohpc_command     for((i=0; i<${#c_ip[@]}; i++)) ; do
% ohpc_command       if ! [[ ${c_ip[i]} =~ ^(([0-9]|[1-9][0-9]|1[0-9][0-9]|2[0-4][0-9]|25[0-5])\.){3}([\
% ohpc_command                                 0-9]|[1-9][0-9]|1[0-9][0-9]|2[0-4][0-9]|25[0-5])$ ]]; then
% ohpc_command         echo "ERROR: Invalid IP address #$i: ${c_ip[i]}"
% ohpc_command         _BADCOUNT=$((_BADCOUNT+1))
% ohpc_command       fi
% ohpc_command       if ! [[ ${c_bmc[i]} =~ ^(([0-9]|[1-9][0-9]|1[0-9][0-9]|2[0-4][0-9]|25[0-5])\.){3}([\
% ohpc_command                                  0-9]|[1-9][0-9]|1[0-9][0-9]|2[0-4][0-9]|25[0-5])$ ]]; then
% ohpc_command         echo "ERROR: Invalid BMC IP address #$i: ${c_bmc[i]}"
% ohpc_command         _BADCOUNT=$((_BADCOUNT+1))
% ohpc_command       fi
% ohpc_command       if ! [[ `echo ${c_mac[i]^^} | egrep "^([0-9A-F]{2}:){5}[0-9A-F]{2}$"` ]]; then
% ohpc_command         echo "ERROR: Invalid MAC address #$i: ${c_mac[i]}"
% ohpc_command         _BADCOUNT=$((_BADCOUNT+1))
% ohpc_command       fi
% ohpc_command     done
% ohpc_command     
% ohpc_command     [[ $_BADCOUNT -eq 0 ]] || exit 3
% ohpc_command 
% ohpc_command     #validateIpInput $nodename_prefix
% ohpc_command }
% ohpc_command 
% ohpc_command function validateHpcInventory() {
% ohpc_command     if [[ -z "${cloudHpcInventory}" || ! -e "${cloudHpcInventory}" ]];then
% ohpc_command       echo "Error: Unable to access Cloud hpc inventory file -> \"${cloudHpcInventory}\""
% ohpc_command       exit 1
% ohpc_command     else
% ohpc_command       . ${cloudHpcInventory} || { echo "Error sourcing ${cloudHpcInventory}"; exit 1; }
% ohpc_command     fi
% ohpc_command     #Verify Cloud IP. Move this to common function validateIpInput
% ohpc_command     _BADCOUNT=0
% ohpc_command     
% ohpc_command     for((i=0; i<${#cc_ip[@]}; i++)) ; do
% ohpc_command       if ! [[ ${cc_ip[i]} =~ ^(([0-9]|[1-9][0-9]|1[0-9][0-9]|2[0-4][0-9]|25[0-5])\.){3}([\
% ohpc_command                                 0-9]|[1-9][0-9]|1[0-9][0-9]|2[0-4][0-9]|25[0-5])$ ]]; then
% ohpc_command         echo "ERROR: Invalid IP address #$i: ${cc_ip[i]}"
% ohpc_command         _BADCOUNT=$((_BADCOUNT+1))
% ohpc_command       fi
% ohpc_command       if ! [[ ${cc_bmc[i]} =~ ^(([0-9]|[1-9][0-9]|1[0-9][0-9]|2[0-4][0-9]|25[0-5])\.){3}([\
% ohpc_command                                  0-9]|[1-9][0-9]|1[0-9][0-9]|2[0-4][0-9]|25[0-5])$ ]]; then
% ohpc_command         echo "ERROR: Invalid BMC IP address #$i: ${cc_bmc[i]}"
% ohpc_command         _BADCOUNT=$((_BADCOUNT+1))
% ohpc_command       fi
% ohpc_command       if ! [[ `echo ${cc_mac[i]^^} | egrep "^([0-9A-F]{2}:){5}[0-9A-F]{2}$"` ]]; then
% ohpc_command         echo "ERROR: Invalid MAC address #$i: ${cc_mac[i]}"
% ohpc_command         _BADCOUNT=$((_BADCOUNT+1))
% ohpc_command       fi
% ohpc_command     done
% ohpc_command     #validateIpInput $cnodename_prefix
% ohpc_command }
% ohpc_command 
% ohpc_command function setup_hosts () {
% ohpc_command     # first search if nodes are already configured. if yes then do not configure again, so that it can be re-run
% ohpc_command     hpc_tag="#### Cloud-HPC nodes entry, entered by HPC Orchestrator ####"
% ohpc_command     #if ! grep -Pq "^$hpc_tag" /etc/hosts; then
% ohpc_command     if ! grep -Fq "$hpc_tag" /etc/hosts; then
% ohpc_command         echo $hpc_tag >> /etc/hosts
% ohpc_command     fi
% ohpc_command     #check if sms_name is already configured, if not then add sms entry
% ohpc_command     if ! grep -Fq ${sms_name} /etc/hosts; then
% ohpc_command         echo -e "${sms_ip}\t${sms_name}" >> /etc/hosts
% ohpc_command     fi
% ohpc_command     for ((i=0; i<$num_ccomputes; i++)) ; do
% ohpc_command        if ! grep -Fq "${cc_name[$i]}" /etc/hosts; then
% ohpc_command            echo -e "${cc_ip[$i]}\t${cc_name[$i]}"
% ohpc_command        fi
% ohpc_command     done >> /etc/hosts
% ohpc_command }
% ohpc_command 
% ohpc_command function setup_cname () {
% ohpc_command     # Determine number of computes and their hostnames
% ohpc_command     export num_computes=${num_computes:-${#c_ip[@]}}
% ohpc_command     for((i=0; i<${num_computes}; i++)) ; do
% ohpc_command        c_name[$i]=${nodename_prefix}$((i+1))
% ohpc_command     done
% ohpc_command     export c_name
% ohpc_command }
% ohpc_command 
% ohpc_command function setup_ccname() {
% ohpc_command     export num_ccomputes=${num_ccomputes:-${#cc_ip[@]}}
% ohpc_command     for((i=0; i<${num_ccomputes}; i++)) ; do
% ohpc_command        cc_name[$i]=${cnodename_prefix}$((i+1))
% ohpc_command     done
% ohpc_command     export cc_name
% ohpc_command }
% ohpc_command 
% ohpc_command function setup_computename() {
% ohpc_command     setup_cname
% ohpc_command     setup_ccname
% ohpc_command }
% ohpc_command 
% ohpc_command function cidr_to_netmask() {
% ohpc_command     cidr=$1
% ohpc_command     value=$(( 0xffffffff ^ ((1 << (32 - $cidr)) - 1) ))
% ohpc_command     netmask="$(( (value >> 24) & 0xff )).$(( (value >> 16) & 0xff )).$(( (value >> 8) & 0xff )).$(( value & 0xff ))"
% ohpc_command     echo $netmask
% ohpc_command }
% ohpc_command 
% ohpc_command function netmask_to_cidr() {
% ohpc_command     nmask=$1
% ohpc_command     # To calculate cidr, we just need to calculate number bits in each octets and add them.
% ohpc_command     cidr_bits=0
% ohpc_command     # iterate through each octets
% ohpc_command     # use dot as saperator
% ohpc_command     IFS=.
% ohpc_command     for octs in $nmask ; do
% ohpc_command        # we can only have 8 combinations in cidr 11111111, 11111110, 11111100, 11110000, 11100000,
% ohpc_command        case $octs in
% ohpc_command           0);;
% ohpc_command           128) cidr_bits=$(($cidr_bits + 1));;
% ohpc_command           192) cidr_bits=$(($cidr_bits + 2));;
% ohpc_command           224) cidr_bits=$(($cidr_bits + 3));;
% ohpc_command           240) cidr_bits=$(($cidr_bits + 4));;
% ohpc_command           248) cidr_bits=$(($cidr_bits + 5));;
% ohpc_command           252) cidr_bits=$(($cidr_bits + 6));;
% ohpc_command           254) cidr_bits=$(($cidr_bits + 7));;
% ohpc_command           255) cidr_bits=$(($cidr_bits + 8));;
% ohpc_command           *) echo "Error: wrong netmask octets $octs";
% ohpc_command        esac
% ohpc_command     done
% ohpc_command     echo $cidr_bits
% ohpc_command }
% ohpc_command 
% ohpc_command function get_ip_from_ipcidr()
% ohpc_command {
% ohpc_command    ipcidr=$1
% ohpc_command    ipadd=$( echo $ipcidr|awk -F '[/]' '{print $1}')
% ohpc_command    echo $ipadd
% ohpc_command }
% ohpc_command function get_netmask_from_ipcidr()
% ohpc_command {
% ohpc_command    ipcidr=$1
% ohpc_command    cidr=$( echo $ipcidr|awk -F '[/]' '{print $2}')
% ohpc_command    netmask="$( cidr_to_netmask $cidr )"
% ohpc_command    echo $netmask
% ohpc_command }
% ohpc_command #  PFILEP
% ohpc_command ## FILE: ./c_init_workaround
% ohpc_command #!/bin/bash
% ohpc_command #set -x
% ohpc_command #Possible requirement for this script: Set hostkey checking to no
% ohpc_command sed --in-place "s|#\s*StrictHostKeyChecking\s*ask|StrictHostKeyChecking no|" /etc/ssh/ssh_config
% ohpc_command 
% ohpc_command #Possible requirement for this script: set ssh key permissions to 600
% ohpc_command #chmod 600 /etc/ssh/ssh_host_ed*_key.pub
% ohpc_command #chmod 600 /etc/ssh/ssh_host_ecdsa_key.pub
% ohpc_command 
% ohpc_command #Copy local cloud_hpc_init to all compute nodes. TODO: Call all nodes through a for loop
% ohpc_command #scp -r /tmp/cloud_hpc_init/ ${cc_ip[0]}:/root
% ohpc_command scp -r /tmp/cloud_hpc_init/ cc1:/root
% ohpc_command 
% ohpc_command #Execute cloud_hpc_init/chpc_init on nodes using pdsh. TODO: Call all nodes through a for loop
% ohpc_command pdsh -w cc1 /root/cloud_hpc_init/chpcInit
% ohpc_command #set +x
% ohpc_command #  PFILEP
% ohpc_command ## FILE: ./setup_cloud_hpc.sh
% ohpc_command #!/bin/bash
% ohpc_command # -----------------------------------------------------------------------------------------
% ohpc_command #  Example Installation Script Template
% ohpc_command #  
% ohpc_command #  This convenience script encapsulates command-line instructions highlighted in
% ohpc_command #  the Install Guide that can be used as a starting point to perform a local
% ohpc_command #  cluster install beginning with bare-metal. Necessary inputs that describe local
% ohpc_command #  hardware characteristics, desired network settings, and other customizations
% ohpc_command #  are controlled via a companion input file that is used to initialize variables 
% ohpc_command #  within this script.
% ohpc_command #   
% ohpc_command #  Please see the Install Guide for more information regarding the
% ohpc_command #  procedure. Note that the section numbering included in this script refers to
% ohpc_command #  corresponding sections from the install guide.
% ohpc_command # -----------------------------------------------------------------------------------------
% ohpc_command 
% ohpc_command if [[ $EUID -ne 0 ]]; then echo "ERROR: Please run $0 as root"; exit 1; fi
% ohpc_command 
% ohpc_command set -E # traps on ERR will now be inherited by shell functions,
% ohpc_command        # command substitution or subshells - equivalent to set -o errtrace
% ohpc_command 
% ohpc_command #For Debugging 
% ohpc_command #set -x
% ohpc_command SCRIPTDIR="$( cd "$( dirname "$( readlink -f "${BASH_SOURCE[0]}" )" )" && pwd -P && echo x)"
% ohpc_command SCRIPTDIR="${SCRIPTDIR%x}"
% ohpc_command cd $SCRIPTDIR
% ohpc_command SCRIPTDIR=$PWD
% ohpc_command pwd
% ohpc_command 
% ohpc_command packstack_install=0
% ohpc_command orchestrator_install=0
% ohpc_command openhpc_install=0
% ohpc_command USECASE=1
% ohpc_command 
% ohpc_command # enable common functions
% ohpc_command source common_functions
% ohpc_command 
% ohpc_command usage () {
% ohpc_command   echo "USAGE: $0 [-f] [-h] [-c] [-o] [-p] [-i=<input.local>] [-n=<cloud_node_inventory>] [-u=<use case id>]"
% ohpc_command   echo " -u,--usecase       Select use case, 1, 2 or 3." 
% ohpc_command   echo " -c,--openhpc       Install OpenHPC using the OpenHPC installation recipe"
% ohpc_command   echo " -f,--forced        Forced run, run all sections with no prompt"
% ohpc_command   echo " -h,--help          Print this message"
% ohpc_command   echo " -i,--input         Location in local inputs"
% ohpc_command   echo " -n,--inventory     Input to cloud HPC inventory file"
% ohpc_command   echo " -o,--orchestrator  Install HPC Orchestrator using the HPC Orchestrator recipe"
% ohpc_command   echo " -p,--packstack     Install OpenStack using the PackStack installation recipe"
% ohpc_command }
% ohpc_command 
% ohpc_command for i in "$@"; do
% ohpc_command   case $i in
% ohpc_command     -c|--openhpc)
% ohpc_command       openhpc_install=1
% ohpc_command       shift # past argument with no value
% ohpc_command     ;;
% ohpc_command     -i=*|--input=*)
% ohpc_command       if echo $i | grep '~'; then
% ohpc_command         echo "ERROR: tilde(~) in pathname not supported."
% ohpc_command         exit 3
% ohpc_command       fi
% ohpc_command       INPUT_LOCAL="${i#*=}"
% ohpc_command       shift # past argument=value
% ohpc_command     ;;
% ohpc_command     -u=*|--usecase=*)
% ohpc_command       export USECASE="${i#*=}"
% ohpc_command 
% ohpc_command       # Check if a valid use case
% ohpc_command       if [[ $USECASE != "1" && $USECASE != "2" && $USECASE != "3" ]]; then
% ohpc_command         echo "Unsupported usecase"
% ohpc_command         exit 1
% ohpc_command       fi
% ohpc_command     ;;
% ohpc_command     -f|--forced)
% ohpc_command       FORCED=YES
% ohpc_command       shift # past argument with no value
% ohpc_command     ;;
% ohpc_command     -n=*|--inventory=*)
% ohpc_command       if echo $i | grep '~'; then
% ohpc_command         echo "ERROR: tilde(~) in pathname not supported."
% ohpc_command         exit 3
% ohpc_command       fi
% ohpc_command       CLOUD_HPC_INVENTORY="${i#*=}"
% ohpc_command       shift # past argument with no value
% ohpc_command     ;;
% ohpc_command     -o|--orchestrator)
% ohpc_command       orchestrator_install=1
% ohpc_command       shift # past argument with no value
% ohpc_command     ;;
% ohpc_command     -p|--packstack)
% ohpc_command       packstack_install=1
% ohpc_command       shift # past argument with no value
% ohpc_command     ;;
% ohpc_command     -h|--help)
% ohpc_command       usage
% ohpc_command       exit 1
% ohpc_command     ;;
% ohpc_command     *)
% ohpc_command       echo "ERROR: Unknown option \"$i\""
% ohpc_command       usage
% ohpc_command       exit 2
% ohpc_command     ;;
% ohpc_command   esac
% ohpc_command done
% ohpc_command 
% ohpc_command # Check if a valid use case is selected
% ohpc_command if [[ $USECASE != "1" && $USECASE != "2" && $USECASE != "3" ]]; then
% ohpc_command   echo "Unsupported usecase, select a valid usecase [1,2,3]"
% ohpc_command   exit 1
% ohpc_command fi
% ohpc_command 
% ohpc_command inputFile=$(readlink -f ${INPUT_LOCAL})
% ohpc_command cloudHpcInventory=$(readlink -f ${CLOUD_HPC_INVENTORY})
% ohpc_command 
% ohpc_command validateInputFile
% ohpc_command validateHpcInventory
% ohpc_command 
% ohpc_command # -------------------------------- Begin Recipe -------------------------------------------
% ohpc_command # Commands below are extracted from the install guide recipe and are intended for 
% ohpc_command # execution on the master SMS host.
% ohpc_command # -----------------------------------------------------------------------------------------
% ohpc_command 
% ohpc_command # Determine number of cloud computes and their hostnames
% ohpc_command setup_computename
% ohpc_command 
% ohpc_command #Set the hostname of the machine
% ohpc_command #hostnamectl set-hostname ${sms_name}
% ohpc_command 
% ohpc_command #Install hpc orchestrator OR openhpc
% ohpc_command if [ "$USECASE" != "3" ]; then
% ohpc_command     if [ "${orchestrator_install}" -eq "1" ]; then
% ohpc_command         mkdir -p /mnt/hpc_orch_iso
% ohpc_command         mount -o loop ${orch_iso_path} /mnt/hpc_orch_iso
% ohpc_command         rpm -Uvh /mnt/hpc_orch_iso/x86_64/Intel_HPC_Orchestrator_release-16.01.002.beta-8.1.x86_64.rpm
% ohpc_command         rpm --import /etc/pki/pgp/HPC-Orchestrator*.asc
% ohpc_command         pushd hpc_cent7/intel
% ohpc_command         time source recipe.sh -f
% ohpc_command         popd
% ohpc_command     fi
% ohpc_command 
% ohpc_command     if [ "${openhpc_install}" -eq "1" ]; then
% ohpc_command         export OHPC_INPUT_LOCAL=$(realpath ${INPUT_LOCAL})
% ohpc_command         pushd hpc_cent7/ohpc
% ohpc_command         time source recipe.sh
% ohpc_command         popd
% ohpc_command     fi
% ohpc_command fi
% ohpc_command 
% ohpc_command #Run packstack install.
% ohpc_command if [ "${packstack_install}" -eq "1" ]; then
% ohpc_command     pushd ../packstack/recipe
% ohpc_command     time source packstack-install.sh -s=${controller_ip} -f=${cc_subnet_cidr} -e=${sms_eth_internal}
% ohpc_command     popd
% ohpc_command fi
% ohpc_command 
% ohpc_command #set up hosts at head node or sms node
% ohpc_command setup_hosts
% ohpc_command #set -x
% ohpc_command case $USECASE in
% ohpc_command   1)
% ohpc_command     pushd 1_combined_controller
% ohpc_command     time source set_os_hpc
% ohpc_command     popd
% ohpc_command   ;;
% ohpc_command   2)
% ohpc_command     pushd 2_cloud_extension
% ohpc_command     time source set_os_hpc
% ohpc_command     popd
% ohpc_command   ;;
% ohpc_command   3)
% ohpc_command     pushd 3_hpc_as_service
% ohpc_command     time source set_os_hpc
% ohpc_command     popd
% ohpc_command   ;;
% ohpc_command   *)
% ohpc_command     echo "ERROR: Unsupported usecase"
% ohpc_command     exit 1
% ohpc_command   ;;
% ohpc_command esac
% ohpc_command     
% ohpc_command true
% ohpc_command 
% ohpc_command #Call sinfo and srun to verify slurm's connection to the compute nodes
% ohpc_command #sinfo
% ohpc_command #srun -N ${num_ccomputes} hostname -i
% ohpc_command 
% ohpc_command # End
% ohpc_command #  PFILEP
% ohpc_command ## FILE: ./teardown_cloud_nodes.sh
% ohpc_command #!/bin/bash
% ohpc_command #This script is designed to be used on our internal sun-hn1 node to clean up all of our configured OpenStack
% ohpc_command #configurations for a clean OpenStack configuration without completely uninstalling and reinstalling the
% ohpc_command #entire OpenStack software stack.
% ohpc_command 
% ohpc_command #Source the keystone file so we have secure access to the OpenStack commands
% ohpc_command source ${HOME}/keystonerc_admin
% ohpc_command 
% ohpc_command #First stop the compute nodes. This is done to cleanly delete the nodes. If you skip the stop step, the
% ohpc_command #nova delete command will sometimes result in an error. This is found to be the safest way to delete
% ohpc_command #nova nodes.
% ohpc_command nova stop cc1
% ohpc_command nova stop cc2
% ohpc_command nova stop cc3
% ohpc_command 
% ohpc_command #Wait for the nova nodes to stop bing in status ACTIVE (i.e. they are now in SHUTOFF)
% ohpc_command nova list | awk {'print $6'} | grep -v 'Status' | grep ACTIVE > /dev/null
% ohpc_command nova_stopped=$?
% ohpc_command until [ "${nova_stopped}" -eq "1" ]; do
% ohpc_command     sleep 5
% ohpc_command     nova list | awk {'print $6'} | grep -v 'Status' | grep ACTIVE > /dev/null
% ohpc_command     nova_stopped=$?
% ohpc_command done
% ohpc_command 
% ohpc_command #Once all the nodes are shutdown, they can safely be deleted from nova.
% ohpc_command nova delete cc1
% ohpc_command nova delete cc2
% ohpc_command nova delete cc3
% ohpc_command 
% ohpc_command #Now that there are no booted nodes and association of a compute node with the ironic nodes,
% ohpc_command #the ironic nodes can safely be deleted.
% ohpc_command ironic node-delete cc1
% ohpc_command ironic node-delete cc2
% ohpc_command ironic node-delete cc3
% ohpc_command 
% ohpc_command #Once the ironic nodes are deleted, we can delete the associated neutron port that was associated with each
% ohpc_command #of the nodes.
% ohpc_command neutron port-delete cc1
% ohpc_command neutron port-delete cc2
% ohpc_command neutron port-delete cc3
% ohpc_command 
% ohpc_command #Now we can delete the shared network that was configured with neutron
% ohpc_command neutron net-delete sharednet1
% ohpc_command 
% ohpc_command #Remove the nova flavor 'baremetal-flavor' association we created with the machine's hardware
% ohpc_command nova flavor-delete baremetal-flavor
% ohpc_command 
% ohpc_command #Remove every image locally saved in glance
% ohpc_command for x in `glance image-list | awk {'print $2'} | grep -v ID`; do
% ohpc_command     glance image-delete $x
% ohpc_command done
% ohpc_command 
% ohpc_command #Finally, remove the keypair association we have in nova. This will leave the system clean and ready for another run
% ohpc_command nova keypair-delete ostack_key
% ohpc_command #  PFILEP
% ohpc_command ## FILE: ./get_cn_mac
% ohpc_command #!/bin/bash
% ohpc_command 
% ohpc_command # check if ipmitool is installed, if not then quit
% ohpc_command 
% ohpc_command #check if BMC_IP, BMC_uname and BMC Password is provided, if not then exi
% ohpc_command 
% ohpc_command function get_bmc_mac {
% ohpc_command     bmc_ip=$1
% ohpc_command     bmc_user=$2
% ohpc_command     bmc_pass=$3
% ohpc_command     if [[ -z $bmc_pass ]]; then
% ohpc_command        ipmi_mac=`ipmitool -E -I lanplus -H $bmc_ip -U $bmc_user lan print 1 |grep "MAC Address"|awk '{print $4}'`
% ohpc_command     else
% ohpc_command        ipmi_mac=`ipmitool -E -I lanplus -H $bmc_ip -U $bmc_user -P $bmc_pass lan print 1 |grep "MAC Address"|awk '{print $4}'`
% ohpc_command     fi
% ohpc_command }
% ohpc_command 
% ohpc_command function get_ipmi_mac_parts {
% ohpc_command     ipmi_mac=$1
% ohpc_command     ipmi_mac_constant_part=${ipmi_mac%:*}
% ohpc_command     ipmi_mac_last_octet=${ipmi_mac##*:}
% ohpc_command }
% ohpc_command 
% ohpc_command function get_comput_mac_octets {
% ohpc_command     ipmi_mac_l=$1
% ohpc_command     hex_ipmi_mac="0x$ipmi_mac_l"
% ohpc_command     mac1_last_octet=$(($hex_ipmi_mac - 2))
% ohpc_command     mac2_last_octet=$(($hex_ipmi_mac - 1))
% ohpc_command     mac1_last_octet=`echo "obase=16; $mac1_last_octet"|bc`
% ohpc_command     mac2_last_octet=`echo "obase=16; $mac2_last_octet"|bc`
% ohpc_command }
% ohpc_command function get_compute_mac {
% ohpc_command     mac1="$ipmi_mac_constant_part:$mac1_last_octet"
% ohpc_command     mac2="$ipmi_mac_constant_part:$mac2_last_octet"
% ohpc_command }
% ohpc_command 
% ohpc_command usage () {
% ohpc_command   echo "USAGE: $0 <bmc_ip> <bmc_user> [bmc_password]"
% ohpc_command }
% ohpc_command 
% ohpc_command ### Main ##
% ohpc_command # check of help is requested
% ohpc_command for i in "$@"; do
% ohpc_command   case $i in
% ohpc_command     -h|--help)
% ohpc_command       usage
% ohpc_command       exit 1
% ohpc_command     ;;
% ohpc_command   esac
% ohpc_command done
% ohpc_command 
% ohpc_command # check if we have at least 3 arguments
% ohpc_command if [[ $# -lt 2 ]]; then
% ohpc_command     echo "Error: Insufficient Arguments"
% ohpc_command     usage
% ohpc_command     exit
% ohpc_command fi
% ohpc_command bmc_ip=$1
% ohpc_command bmc_user=$2
% ohpc_command bmc_pass=$3
% ohpc_command get_bmc_mac $bmc_ip $bmc_user $bmc_pass 
% ohpc_command # check if we got the virtual MAC
% ohpc_command if [[ -z $ipmi_mac ]]; then
% ohpc_command     echo "Error: BMC Communication Error"
% ohpc_command     exit
% ohpc_command fi
% ohpc_command get_ipmi_mac_parts $ipmi_mac
% ohpc_command get_comput_mac_octets $ipmi_mac_last_octet
% ohpc_command get_compute_mac
% ohpc_command echo "Compute MAC1: $mac1"
% ohpc_command echo "Compute MAC2: $mac2"
% end_ohpc_run
