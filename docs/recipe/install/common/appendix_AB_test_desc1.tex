This appendix details the installation and basic use of the integration test suite used to support the releases. This suite is not intended to replace the validation performed by component development teams, but is instead, devised to confirm component builds are functional and interoperable within the modular OpenHPC environment. The test suite depends on OpenHPC test suite test-suite-chpc but adds two additional tests 'ostack' and 'dib'. The OpenHPC test suite is generally organized by components and the CI workflow relies on running the full suite using Jenkins to test multiple OS configurations and installation recipes. To facilitate customization and running of the test suite locally, we provide these tests in a standalone RPM. Please install test rpm at controller node and then execute "hpc\_in\_oscloud-tests" script to run the test. This script run tests at controller node, copies test rpm  and its dependencies to sms node and install same tests at sms node. Then script executes tests at sms node. Altenratively you can install test\_suite-chpc directly on sms node along with its dependenciesi (perl-Test-Harness, perl-XML-Generator, test-suite-ohpc). Since the sms node is not directly connected to a public network, you need to download the RPM at the controller node and copy them manually to the sms node. \\
Test suite depends on automake-ohpc and autoconf-ohpc packages. Before installing test suite on controller node please enable an OpenHPC repository locally through installation of an ohpc-release RPM. For installing OpenHPC repository, please refer to http://openhpc.community.

