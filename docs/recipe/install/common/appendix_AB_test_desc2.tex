The RPM installation creates a user named ohpc-test to house the test suite and provide an isolated environment for execution. Configuration of the test suite is done using standard GNU autotools semantics and the BATS shell-testing framework is used to execute and log a number of individual unit tests. Some tests require privileged execution, so a different combination of tests will be enabled depending on which user executes the top-level configure script. Non-privileged tests requiring execution on one or more compute nodes are submitted as jobs through the HPC resource manager. The tests are further divided into "short" and "long" run categories. The short run configuration is a subset of approximately 180 tests to demonstrate basic functionality of key components (e.g. MPI stacks) and should complete in 10-20 minutes. The long run (around 1000 tests) is comprehensive and can take an hour or more to complete. Most components can be tested individually, but a default configuration is setup to enable collective testing. To test an isolated component, use the configure option to disable all tests, then re-enable the desired test to run. Before running test directly on sms node, run script "enable-ostack-tests.sh" to enable cloud hpc tests, followed by "./bootstrap" and "./configure". The --help option to configure will display all possible tests. Example output is shown below (some output is omitted for the sake of brevity).

