\renewcommand\thesubsection{\Alph{subsection}}
\subsection{Installation Template}

This appendix highlights the availability of a companion installation script that is included with documentation. This script, when combined with local site inputs, can be used to implement a starting
recipe for bare-metal system installation and configuration. This template script is used during validation efforts to test cluster installations and is provided as a convenience for administrators as a starting point for potential site customization.
\begin{center}
	\begin{tcolorbox}[]
		\small
Note that the template script provided is intended for use during initial installation and is not designed for
repeated execution. If modifications are required after using the script initially, we recommend running the
relevant subset of commands interactively.
\end{tcolorbox}
\end{center}
The template script relies on the use of a pair of simple text files to define local site variables that were outlined
in Section 1.3. These files are called by the 'setup\_cloud\_hpc.sh' script in the default install directory, '/opt/ohpc/pub/doc/recipes'.
The template install script is intended for execution on the controller host and is installed as part of the docs-chpc package. After enabling the repository and reviewing the guide for additional information on the intent of the commands, the general starting approach for using this template
is as follows:


1. Install the docs-chpc and dib-chpc packages on controller node, both rpms are part of tar ball distribution

\begin{lstlisting}[language=bash,keywords={},upquote=true]
[controller]# yum -y install docs-chpc*
[controller]# yum -y install dib-chpc*
\end{lstlisting}

2. Copy the provided template input file to use as a starting point to define local site settings. Here we assume two input files. *input.local file very similar to OpenHPC input.local file, we kept this format as is for cloud node instantiation and for cloud node configuration we used "*.INVENTORY" file. The main reason for having two configuration files is to take care of cloud-burst scenario, where "input.local" file represents configuration of HPC system and "INVENTORY" file represents configuration of OpenStack cloud nodes. For "HPC as a Service" we kept same format and node entries in input.local file is ignored by recipe of "HPC as  a service".

\begin{lstlisting}[language=bash,keywords={},upquote=true]
[controller]# cp /opt/ohpc/pub/doc/recipes/Template-INPUT.LOCAL ~/input.local
[controller]# cp /opt/ohpc/pub/doc/recipes/Template-INVENTORY ~/inventory
\end{lstlisting}

3. Update input.local with sms node name and pointer to OpenHPC repo. Update inventory file with information about controller node, cloud nodes and images.

4. Run the installation script:
% see rest of the detail in appendix_AB_example_<resource Manager>.tex file
