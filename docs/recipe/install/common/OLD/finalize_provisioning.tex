\subsection{Finalizing provisioning configuration} \label{sec:assemble_bootstrap}

\Warewulf{} employs a two-stage boot process for provisioning nodes via
creation of a bootstrap image that is used to initialize the process, and a virtual node
file system capsule containing the full system image. This section highlights
creation of the necessary provisioning images, followed by the registration of
desired compute nodes.

\subsubsection{Assemble bootstrap image}

The bootstrap image includes the runtime kernel and associated modules, as well
as some simple scripts to complete the provisioning process. The
following commands highlight the inclusion of additional drivers and creation
of the bootstrap image based on the running kernel.

% begin_ohpc_run
% ohpc_comment_header Assemble bootstrap image \ref{sec:assemble_bootstrap}
\begin{lstlisting}[language=bash,literate={-}{-}1,keywords={},upquote=true]
# (Optional) Include Lustre drivers; needed if enabling Lustre client on computes
[sms](*\#*) export WW_CONF=/etc/warewulf/bootstrap.conf
[sms](*\#*) echo "drivers += updates/kernel/" >> $WW_CONF

# Build bootstrap image
[sms](*\#*) wwbootstrap `uname -r`
\end{lstlisting}
% end_ohpc_run

\subsubsection{Assemble Virtual Node File System (VNFS) image}

With the local site customizations in place, the following step uses the
\texttt{wwvnfs} command to assemble a VNFS capsule from the chroot environment
defined for the {\em compute} instance. 

% begin_ohpc_run
% ohpc_validation_comment Assemble VNFS
\begin{lstlisting}[language=bash,literate={-}{-}1,keywords={},upquote=true]
[sms](*\#*) wwvnfs --chroot $CHROOT
\end{lstlisting}
% end_ohpc_run

\subsubsection{Register nodes for provisioning}

In preparation for provisioning, we can now define the desired network settings
for four example compute nodes with the underlying provisioning system and
restart the \texttt{dhcp} service. Note the use of variable names for the
desired compute hostnames, node IPs, and MAC addresses which should be modified
to accommodate local settings and hardware.  By default, \Warewulf{} uses
network interface names of the \texttt{eth\#} variety and adds kernel boot
arguments to maintain this scheme on newer kernels. Consequently, when specifying
the desired provisioning interface via the \texttt{\$eth\_provision} variable,
it should follow this convention. Alternatively, if you prefer to use the
predictable network interface naming scheme (e.g. names like \texttt{en4s0f0}),
additional steps are included to alter the default kernel boot arguments and take
the \texttt{eth\#} named interface down after bootrapping so the normal init
process can bring it up again using the desired name.

Also included in these steps are commands
to enable \Warewulf{} to manage IPoIB settings and corresponding definitions of
IPoIB addresses for the compute nodes. This is typically optional unless you
are planning to include a \Lustre{} client mount over \InfiniBand{}. The final step
in this process associates the VNFS image assembled in previous steps with the
newly defined compute nodes, utilizing the user credential files and munge key
that were imported in \S\ref{sec:file_import}.

%%% % begin_ohpc_run
%%% % ohpc_validation_comment Add hosts to cluster
%%% 
%%% \begin{lstlisting}[language=bash,keywords={},upquote=true,basicstyle=\footnotesize\ttfamily,]
%%% # Define four compute nodes and network settings 
%%% [sms](*\#*) wwsh -y node new ${c_name[0]} --ipaddr=${c_ip[0]} --hwaddr=${c_mac[0]} -D ${eth_provision}
%%% [sms](*\#*) wwsh -y node new ${c_name[1]} --ipaddr=${c_ip[1]} --hwaddr=${c_mac[1]} -D ${eth_provision}
%%% [sms](*\#*) wwsh -y node new ${c_name[2]} --ipaddr=${c_ip[2]} --hwaddr=${c_mac[2]} -D ${eth_provision}
%%% [sms](*\#*) wwsh -y node new ${c_name[3]} --ipaddr=${c_ip[3]} --hwaddr=${c_mac[3]} -D ${eth_provision}
%%% 
%%% # Optionally register additional compute nodes
%%% [sms](*\#*) for ((i=4; i<$num_computes; i++)) ; do
%%%               wwsh -y node new ${c_name[$i]} --ipaddr=${c_ip[$i]} --hwaddr=${c_mac[$i]} -D ${eth_provision}
%%%            done
%%% 
%%% # Define provisioning image for hosts
%%% [sms](*\#*) wwsh -y provision set "${compute_regex}" --vnfs=centos7.1 --bootstrap=`uname -r` \
%%%      --files=dynamic_hosts,passwd,group,shadow,munge.key 
%%% 
%%% # Optionally define IPoIB network settings (required if planning to mount Lustre over IB)
%%% [sms](*\#*) for ((i=0; i<$num_computes; i++)) ; do
%%%               wwsh -y node set ${c_name[$i]} -D ib0 --ipaddr=${c_ipoib[$i]} --netmask=${ipoib_netmask}
%%%            done
%%% [sms](*\#*) wwsh -y provision set "${compute_regex}" --fileadd=ifcfg-ib0.ww
%%% 
%%% # Restart dhcp / update PXE
%%% [sms](*\#*) sytemctl restart dhcpd
%%% [sms](*\#*) wwsh pxe update
%%% \end{lstlisting}
%%% % end_ohpc_run

