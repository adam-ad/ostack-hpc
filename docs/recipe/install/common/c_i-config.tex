In previous section we created template for cloud-init for hpc head node and hpc compute nodes. We need to update these template with user defined inputs like IP Address, node names. With these updates, cloud-init script is ready to deploy with OpenStack Nova.

Copy cloud-init template to working folder

\begin{lstlisting}[language=bash,keywords={}]
chpcInitPath=/opt/ohpc/admin/cloud_hpc_init
# if directory exists then mv to Old directory. TBD
mkdir -p $chpcInitPath
#copy Cloud HPC files to temp working directory
sudo cp -fr -L < ${SCRIPTDIR} >/ cloud_hpc_init/${chpc_base}/* $chpcInitPath/
export chpcInit=$chpcInitPath/chpc_init
export chpcSMSInit=$chpcInitPath/chpc_sms_init
\end{lstlisting}


Update sms\_ip in compute node cloud-init template with HPC head node. 

\begin{lstlisting}[language=bash,keywords={}]
sudo sed -i -e "s/<sms_ip>/${sms_ip}/g" $chpcInit
\end{lstlisting}


Update HPC head node cloud-init template with compute name prefix as defined by user

\begin{lstlisting}[language=bash,keywords={}]
sudo sed -i -e "s/<update_cnodename_prefix>/${cnodename_prefix}/g" $chpcSMSInit
sudo sed -i -e "s/<update_num_ccomputes>/${num_ccomputes}/g" $chpcSMSInit
Update hostname of HPC head node & NTP server information
sudo sed -i -e "s/<update_ntp_server>/${controller_ip}/g" $chpcSMSInit
sudo sed -i -e "s/<update_sms_name>/${sms_name}/g" $chpcSMSInit
\end{lstlisting}

Optionally if user enabled mrsh or clusteshell, then update cloud-init accordingly

\begin{lstlisting}[language=bash,keywords={}]
if [[ ${enable_mrsh} -eq 1 ]];then
   # update mrsh for sms node
   cat $CHPC_SCRIPTDIR/sms/update_mrsh >> $chpcSMSInit
fi
if [[ ${enable_clustershell} -eq 1 ]];then
   # update clustershell for sms node
   cat $CHPC_SCRIPTDIR/sms/update_clustershell >> $chpcSMSInit
fi
\end{lstlisting}
