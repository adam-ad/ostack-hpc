The term "HPC as a Service" refers to an on demand instantiation of an HPC service in a cloud environment. This guide presents a simple "HPC cluster" instantiation procedure on an existing OpenStack (Mitaka) system. "HPC as a service" relies on two main principals to instantiate the HPC service:


\begin{list}{}
	\item 	1. Providing pre-built OS images for compute nodes with HPC optimized software.
	\item   2. Use of cloud-init to configure and tune HPC services.
	\item   
	 
\end{list}
	
This document provides a simple guide to build HPC optimized OS images, prepare cloud-init recipes and finally instantiate a fully functional HPC System using those images and cloud-init. 

Chapter 2-5 of this documents are dedicated to manual installation and configuration of HPC in an OpenStack based cloud. All the manual instructions mentioned in these chapters (2-5) are captured into an automated recipe which is installed at "/opt/ohpc/pub/doc/recipes/centos7/x86\_64/openstack/SLURM" location. This generated recipe is used for continuous integration testing. 
Appendix A provides instruction on how to use auto generated recipe. For first time users, it is recommended to use auto generated recipe and then modify according to their site requirement.

Recipes will instantiate a baremetal HPC head node (a.k.a. sms node) and baremetal HPC compute nodes (a.k.a CN) using pre-configured OpenStack images. The terms "head" and "sms" are used interchangeably in this guide.

OS Images are built using components from the OpenHPC software stack. OpenHPC represents an aggregation of a number of common ingredients required to deploy and manage an HPC Linux* cluster including resource management, I/O clients, development tools, and a variety of scientific libraries. These packages have been pre-built with HPC integration in mind using a mix of open-source components. The documentation herein is intended to be reasonably generic,
but uses the underlying motivation of a small, 4-node state-full cluster installation to define a step-by-step process. 

Several optional customizations are included and the intent is that these collective instructions can be modified as needed for local site customizations.

