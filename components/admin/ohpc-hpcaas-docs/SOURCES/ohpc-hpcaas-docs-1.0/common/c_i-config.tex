In previous section we created template for cloud-init for hpc head node and hpc compute nodes. We need to update these template with user defined inputs like IP Address, node names. With these updates, cloud-init script is ready to deploy with OpenStack Nova.

Copy cloud-init template to working folder

% begin_ohpc_run
% ohpc_validation_newline
% ohpc_validation_comment ## ZFILEZ
% ohpc_validation_comment ## XFILEX
% ohpc_command #!/bin/bash
% ohpc_validation_comment # FILE: prepare_cloud_init

\begin{lstlisting}[language=bash,keywords={}]
[ctrlr](*\#*) chpcInitPath=/opt/ohpc/admin/cloud_hpc_init
[ctrlr](*\#*) # if directory exists then mv to Old directory. TBD
[ctrlr](*\#*) mkdir -p $chpcInitPath
[ctrlr](*\#*) #copy Cloud HPC files to temp working directory
[ctrlr](*\#*) sudo cp -fr -L < ${SCRIPTDIR} >/ cloud_hpc_init/${chpc_base}/* $chpcInitPath/
[ctrlr](*\#*) export chpcInit=$chpcInitPath/chpc_init
[ctrlr](*\#*) export chpcSMSInit=$chpcInitPath/chpc_sms_init
\end{lstlisting}
% end_ohpc_run

Update sms\_ip in compute node cloud-init template with HPC head node. 
% begin_ohpc_run
\begin{lstlisting}[language=bash,keywords={}]
[ctrlr](*\#*) sudo sed -i -e "s/<sms_ip>/${sms_ip}/g" $chpcInit
\end{lstlisting}
% end_ohpc_run

Update HPC head node cloud-init template with compute name prefix as defined by user

% begin_ohpc_run
\begin{lstlisting}[language=bash,keywords={}]
[ctrlr](*\#*) sudo sed -i -e "s/<update_cnodename_prefix>/${cnodename_prefix}/g" $chpcSMSInit
[ctrlr](*\#*) sudo sed -i -e "s/<update_num_ccomputes>/${num_ccomputes}/g" $chpcSMSInit
[ctrlr](*\#*) # Update hostname of HPC head node & NTP server information
[ctrlr](*\#*) sudo sed -i -e "s/<update_ntp_server>/${controller_ip}/g" $chpcSMSInit
[ctrlr](*\#*) sudo sed -i -e "s/<update_sms_name>/${sms_name}/g" $chpcSMSInit
\end{lstlisting}
% end_ohpc_run

Optionally if user enabled mrsh or clustershell, then update cloud-init accordingly

% begin_ohpc_run
\begin{lstlisting}[language=bash,keywords={}]
[ctrlr](*\#*) if [[ ${enable_mrsh} -eq 1 ]];then
[ctrlr](*\#*)    # update mrsh for sms node
[ctrlr](*\#*)    cat $CHPC_SCRIPTDIR/sms/update_mrsh >> $chpcSMSInit
[ctrlr](*\#*) fi
[ctrlr](*\#*) if [[ ${enable_clustershell} -eq 1 ]];then
[ctrlr](*\#*)    # update clustershell for sms node
[ctrlr](*\#*)    cat $CHPC_SCRIPTDIR/sms/update_clustershell >> $chpcSMSInit
[ctrlr](*\#*) fi
\end{lstlisting}
% end_ophc_run
