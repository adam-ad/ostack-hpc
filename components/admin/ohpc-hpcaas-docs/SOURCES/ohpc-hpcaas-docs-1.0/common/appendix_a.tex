% begin_ohpc_run
% ohpc_validation_comment   XFILEX
% ohpc_validation_comment #!/bin/bash
% ohpc_validation_comment # FILE:  prepare_chpc_openstack ---------------------------------------------------------------------------------
% ohpc_validation_comment This script installs and configures ironic for baremetal provisioning on CentOS 7
% ohpc_validation_comment Using the OpenStack-Mitaka release.
% ohpc_validation_comment This relies on the packstack installation to happen first and the keystonerc_admin
% ohpc_validation_comment file being in the user's home directory. It is assumed this script is run with
% ohpc_validation_comment sudo permissions.
% ohpc_validation_comment ---------------------------------------------------------------------------------
% ohpc_validation_comment Set SELinux to permissive
% ohpc_validation_comment 
\begin{lstlisting}[language=bash,keywords={}]
[ctrlr](*\#*) #
[ctrlr](*\#*) setenforce 0
[ctrlr](*\#*) #
\end{lstlisting}
% ohpc_validation_comment 
% ohpc_validation_comment  
% ohpc_validation_comment Source the keystonerc_admin file
% ohpc_validation_comment 
\begin{lstlisting}[language=bash,keywords={}]
[ctrlr](*\#*) #
[ctrlr](*\#*) source ${HOME}/keystonerc_admin
[ctrlr](*\#*) #
\end{lstlisting}
% ohpc_validation_comment 
% ohpc_validation_comment  
% ohpc_validation_comment #Create roles for the baremetal service. These can be used later to give special permissions to baremetal. This script will be using the defaults.
% ohpc_validation_comment 
\begin{lstlisting}[language=bash,keywords={}][ctrlr](*\#*) openstack role list | grep -i baremetal_admin
[ctrlr](*\#*) #
[ctrlr](*\#*) role_exists=$?
[ctrlr](*\#*) if [ "${role_exists}" -ne "0" ]; then 
[ctrlr](*\#*)     openstack role create baremetal_admin
[ctrlr](*\#*) fi
[ctrlr](*\#*) 
[ctrlr](*\#*) openstack role list | grep -i baremetal_observer
[ctrlr](*\#*) role_exists=$?
[ctrlr](*\#*) if [ "${role_exists}" -ne "0" ]; then 
[ctrlr](*\#*)     openstack role create baremetal_observer
[ctrlr](*\#*) fi
[ctrlr](*\#*) systemctl restart openstack-ironic-api
[ctrlr](*\#*) 
[ctrlr](*\#*) #
\end{lstlisting}
% ohpc_validation_comment 
% ohpc_validation_comment 
% ohpc_validation_comment #Ensure the utilities for baremetal are installed
% ohpc_validation_comment 
\begin{lstlisting}[language=bash,keywords={}]
[ctrlr](*\#*) #
[ctrlr](*\#*) yum install -y tftp-server syslinux-tftpboot xinetd
[ctrlr](*\#*) #
\end{lstlisting}
% ohpc_validation_comment 
% ohpc_validation_comment 
% ohpc_validation_comment #Make the directory for tftp and give it the ironic owner
% ohpc_validation_comment 
\begin{lstlisting}[language=bash,keywords={}]
[ctrlr](*\#*) #
[ctrlr](*\#*) mkdir -p /tftpboot
[ctrlr](*\#*) chown -R ironic /tftpboot
[ctrlr](*\#*) #
\end{lstlisting}
% ohpc_validation_comment 
% ohpc_validation_comment 
% ohpc_validation_comment #Configure /etc/xinet.d/tftp
% ohpc_validation_comment 
\begin{lstlisting}[language=bash,keywords={}]
[ctrlr](*\#*) #
[ctrlr](*\#*) echo "service tftp" > /etc/xinetd.d/tftp
[ctrlr](*\#*) echo "{" >> /etc/xinetd.d/tftp
[ctrlr](*\#*) echo "  protocol        = udp" >> /etc/xinetd.d/tftp
[ctrlr](*\#*) echo "  port            = 69" >> /etc/xinetd.d/tftp
[ctrlr](*\#*) echo "  socket_type     = dgram" >> /etc/xinetd.d/tftp
[ctrlr](*\#*) echo "  wait            = yes" >> /etc/xinetd.d/tftp
[ctrlr](*\#*) echo "  user            = root" >> /etc/xinetd.d/tftp
[ctrlr](*\#*) echo "  server          = /usr/sbin/in.tftpd" >> /etc/xinetd.d/tftp
[ctrlr](*\#*) echo "  server_args     = -v -v -v -v -v --map-file /tftpboot/map-file /tftpboot" >> /etc/xinetd.d/tftp
[ctrlr](*\#*) echo "  disable         = no" >> /etc/xinetd.d/tftp
[ctrlr](*\#*) echo "  # This is a workaround for Fedora, where TFTP will listen only on" >> /etc/xinetd.d/tftp
[ctrlr](*\#*) echo "  # IPv6 endpoint, if IPv4 flag is not used." >> /etc/xinetd.d/tftp
[ctrlr](*\#*) echo "  flags           = IPv4" >> /etc/xinetd.d/tftp
[ctrlr](*\#*) echo "}" >> /etc/xinetd.d/tftp
[ctrlr](*\#*) #
\end{lstlisting}
% ohpc_validation_comment 
% ohpc_validation_comment 
% ohpc_validation_comment #Restart the xinetd service
% ohpc_validation_comment 
\begin{lstlisting}[language=bash,keywords={}]
[ctrlr](*\#*) #
[ctrlr](*\#*) systemctl restart xinetd
[ctrlr](*\#*) #
\end{lstlisting}
% ohpc_validation_comment 
% ohpc_validation_comment 
% ohpc_validation_comment #Copy the PXE linux files to the tftpboot directory we created
% ohpc_validation_comment 
\begin{lstlisting}[language=bash,keywords={}]
[ctrlr](*\#*) #
[ctrlr](*\#*) cp /var/lib/tftpboot/pxelinux.0 /tftpboot
[ctrlr](*\#*) cp /var/lib/tftpboot/chain.c32 /tftpboot
[ctrlr](*\#*) #
\end{lstlisting}
% ohpc_validation_comment 
% ohpc_validation_comment 
% ohpc_validation_comment #Generate a map file for the PXE files
% ohpc_validation_comment 
\begin{lstlisting}[language=bash,keywords={}]
[ctrlr](*\#*) #
[ctrlr](*\#*) echo 're ^(/tftpboot/) /tftpboot/\2' > /tftpboot/map-file
[ctrlr](*\#*) echo 're ^/tftpboot/ /tftpboot/' >> /tftpboot/map-file
[ctrlr](*\#*) echo 're ^(^/) /tftpboot/\1' >> /tftpboot/map-file
[ctrlr](*\#*) echo 're ^([^/]) /tftpboot/\1' >> /tftpboot/map-file
[ctrlr](*\#*) #
\end{lstlisting}
% ohpc_validation_comment 
% ohpc_validation_comment 
% ohpc_validation_comment #Edit /etc/ironic/ironic.conf file with the <controller_ip> variable's value
% ohpc_validation_comment 
\begin{lstlisting}[language=bash,keywords={}]
[ctrlr](*\#*) #
[ctrlr](*\#*) sed --in-place "s|#tftp_server=\$my_ip|tftp_server=${controller_ip}|" /etc/ironic/ironic.conf
[ctrlr](*\#*) sed --in-place "s|#tftp_root=/tftpboot|tftp_root=/tftpboot|" /etc/ironic/ironic.conf
[ctrlr](*\#*) sed --in-place "s|#ip_version=4|ip_version=4|" /etc/ironic/ironic.conf
[ctrlr](*\#*) sed --in-place "s|#automated_clean=true|automated_clean=false|" /etc/ironic/ironic.conf
[ctrlr](*\#*) #
\end{lstlisting}
% ohpc_validation_comment 
% ohpc_validation_comment 
% ohpc_validation_comment #Edit /etc/nova/nova.conf file
% ohpc_validation_comment 
\begin{lstlisting}[language=bash,keywords={}]
[ctrlr](*\#*) #
[ctrlr](*\#*) sed --in-place "s|reserved_host_memory_mb=512|reserved_host_memory_mb=0|" 
/etc/nova/nova.conf
[ctrlr](*\#*) sed --in-place "s|#scheduler_host_subset_size=1|scheduler_host_subset_size=9999999|" /etc/nova/nova.conf
[ctrlr](*\#*) sed --in-place "s|#scheduler_use_baremetal_filters=false|scheduler_use_baremetal_filters=true|" /etc/nova/nova.conf
[ctrlr](*\#*) #
\end{lstlisting}
% ohpc_validation_comment 
% ohpc_validation_comment 
% ohpc_validation_comment # Enable meta data
% ohpc_validation_comment # Edit /etc/neutron/dhcp_agent.ini
% ohpc_validation_comment 
\begin{lstlisting}[language=bash,keywords={}]
[ctrlr](*\#*) #
[ctrlr](*\#*) sed --in-place "s|enable_isolated_metadata\ =\ False|enable_isolated_metadata\ =\ True|" /etc/neutron/dhcp_agent.ini
[ctrlr](*\#*) sed --in-place "s|#force_metadata\ =\ false|force_metadata\ =\ True|" /etc/neutron/dhcp_agent.ini
[ctrlr](*\#*) #
\end{lstlisting}
% ohpc_validation_comment 
% ohpc_validation_comment  
% ohpc_validation_comment #####
% ohpc_validation_comment # Enable internal dns for hostname resolution, if it already not set
% ohpc_validation_comment # manipulating configuration file via shell, alternate is to use openstack-config (TODO)
% ohpc_validation_comment ####
% ohpc_validation_comment # setup dns domain first
% ohpc_validation_comment 
\begin{lstlisting}[language=bash,keywords={}]
[ctrlr](*\#*) #
[ctrlr](*\#*) if grep -q "^dns_domain.*openstacklocal$" /etc/neutron/neutron.conf; then
[ctrlr](*\#*)    sed -in-place  "s|^dns_domain.*|dns_domain = oslocal|" /etc/neutron/neutron.conf
[ctrlr](*\#*) #
[ctrlr](*\#*) else
[ctrlr](*\#*) #
[ctrlr](*\#*)    if ! grep -q "^dns_domain" neutron.conf; then
[ctrlr](*\#*)        sed -in-place  "s|^#dns_domain = openstacklocal$|dns_domain = oslocal|" /etc/neutron/neutron.conf
[ctrlr](*\#*) #
[ctrlr](*\#*)    fi
[ctrlr](*\#*) fi
[ctrlr](*\#*) #
\end{lstlisting}
% ohpc_validation_comment 
% ohpc_validation_comment # configure ml2 dns driver for neutron
% ohpc_validation_comment 
\begin{lstlisting}[language=bash,keywords={}]
[ctrlr](*\#*) #
[ctrlr](*\#*) ml2file=/etc/neutron/plugins/ml2/ml2_conf.ini
[ctrlr](*\#*) #
[ctrlr](*\#*) if ! grep -q "^extension_drivers" $ml2file; then
[ctrlr](*\#*)     # Assuming there is a place holder in comments, replace that string
[ctrlr](*\#*)     sed -in-place  "s|^#extension_drivers.*|extension_drivers = port_security,dns|" $ml2file
[ctrlr](*\#*) #
[ctrlr](*\#*) else
[ctrlr](*\#*) #
[ctrlr](*\#*)     # Entry is present, check if dns is already present, if not then enable
[ctrlr](*\#*) #
[ctrlr](*\#*)     if ! grep "^extension_drivers" $ml2file|grep -q dns; then
[ctrlr](*\#*)         current_dns=`grep "^extension_drivers" $ml2file`
[ctrlr](*\#*)         new_dns="$current_dns,dns"
[ctrlr](*\#*)         sed -in-place  "s|^extension_drivers.*|$new_dns|" $ml2file
[ctrlr](*\#*) #
[ctrlr](*\#*)     fi 
[ctrlr](*\#*) #
[ctrlr](*\#*) fi
[ctrlr](*\#*) #
\end{lstlisting}
% ohpc_validation_comment 
% ohpc_validation_comment #------
% ohpc_validation_comment 
% ohpc_validation_comment # Now restart the services
% ohpc_validation_comment #Restart ironic, nova, neutron, and ovs to load in the new configuration
\begin{lstlisting}[language=bash,keywords={}]
[ctrlr](*\#*) #
[ctrlr](*\#*) systemctl restart neutron-dhcp-agent
[ctrlr](*\#*) systemctl restart neutron-openvswitch-agent
[ctrlr](*\#*) systemctl restart neutron-metadata-agent
[ctrlr](*\#*) systemctl restart neutron-server
[ctrlr](*\#*) systemctl restart openstack-nova-scheduler
[ctrlr](*\#*) systemctl restart openstack-nova-compute
[ctrlr](*\#*) systemctl restart openstack-ironic-conductor
[ctrlr](*\#*) #
\end{lstlisting}
% ohpc_validation_comment 
% end_ohpc_run